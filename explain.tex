\documentclass{jarticle}
%
\setlength{\topmargin}{-1.04cm}%例:上余白を設定
\setlength{\oddsidemargin}{-0.54cm}%例:左余白を1.5cmにする
\setlength{\evensidemargin}{-0.54cm}%例:左余白を1.5cmにする
\setlength{\textwidth}{18cm}%例:一行の幅を18cmにする
\setlength{\textheight}{25cm}%例:一ページの文章の縦の長さを25cmにする
%
\usepackage{amsmath}
\usepackage{multicol}
\title{初めてgitを使うものです。宜しくお願いします。}
\author{坂本悠作}
\date{\today}
\begin{document}
\maketitle
\newpage

\begin{multicols}{2}
\section{日本語入力を設定する}
nkf --overwrite -w8 test.tex\\
\section{コンパイルの方法}
platex test.tex\\
xdvi test.dvi\\
dvipdfmx test.dvi\\
evince text.pdf
\section{改行の方法}
円マークを\underline{二個}書くと改行されます。
\section{ラインを引きます}
\begin{tabular}{|r||c|c|}
\hline
\textbf{都道府県}&\textbf{名所}\\
\hline
滋賀&琵琶湖\\
\cline{2-2}
&アグリパーク龍王\\
\hline
\end{tabular}
\\
\vspace{10mm}
\\
\begin{tabular}{|l|l|l|}
\hline
\multicolumn{3}{|c|}{\textbf{都道府県と所在地}}\\
\hline
\multicolumn{1}{|c||}{\textbf{都道府県}}&\multicolumn{1}{c|}{\textbf{県庁所在地}}&\multicolumn{1}{c|}{\textbf{所在地方}}\\
\hline
\end{tabular}

\section{数式}
$a^{2}$とかが簡単にかけます\\
\begin{equation}
y=a^2+ax+b
\end{equation}
という式番号も自動でつけてくれます。\\
\begin{equation}
\frac{3}{6}
\end{equation}
という感じデス。
\[
y=\left(\frac{x^2_p}{y}\right)
\]\\
$x_{p}$みたいな添字についても簡単にかける!\\
\S\\
\pounds\\
\o\\
\O\\
\l\\
\ss\\
$\Omega$\\
$\omega$\\
$\psi$\\
%でコメントアウトですかね
\end{multicols}

\end{document}
